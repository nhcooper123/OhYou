\documentclass[a4paper,12pt]{article}

\usepackage[osf]{mathpazo} % palatino
\usepackage{ms}            % load the template
\usepackage[round]{natbib} % author-year citations
\usepackage{parskip}  
\usepackage{longtable}
\usepackage{pdflscape}
\usepackage{fullpage}  
\pagenumbering{arabic}    

\begin{document}

\raggedright
\doublespacing
\setlength{\parindent}{1cm}

\section{Supporting Information}
\setcounter{table}{0} \renewcommand{\thetable}{S\arabic{table}}

  \subsection{Literature review methods}
    To get an overview of the use of OU models in ecology, evolution and palaeontology, we used Google Scholar (accessed 13th March 2015) to locate papers published between 2005 (when the R package ouch was released; Butler and King 2004) and 2014 that contained the terms ``Ornstein Uhlenbeck'' and either ``'ecology'', or ``evolution'' and ``biology'' (the ``biology'' term was added to omit physics papers which also use the term ``evolution''), or ``paleo/palaeo''. 
    We also recorded the total number of papers containing the terms ``ecology'', or ``evolution'' and ``biology'', or ``paleo/palaeo'', published between 2005 and 2014 and plotted the number of OU papers published each year as a proportion of the total number of papers published (Fig. 1 in the main text). 

    Next we filtered our Google Scholar search results to focus on empirical papers using OU models (rather than pure methods papers) published in the following journals: The American Naturalist, Ecology Letters, Evolution, Journal of Evolutionary Biology, Nature, Proceedings of the National Academy of Sciences, USA, Proceedings of the Royal Society B: Biological Sciences, and Science.
    We only include papers up to the end of 2013 to ensure completeness.

    For each of these papers we recorded the number of species in the analysis, the study group (amphibians, birds, fish, mammals, reptiles, invertebrates or plants), the statistical package or specific R package used to fit the models, and the reason the authors state for using an OU model (ancestral state reconstructions, detecting convergent evolution, controlling for phylogeny, selecting a model of trait evolution, or other). 
    Where papers included multiple analyses using different numbers of species we used the median number of taxa. 
    Where papers had multiple study groups, statistics/R packages or reasons for fitting OU models we counted them in each relevant category. 
    We summarise these results in Figures 1 and 2 and Table 1 in the main text, and the full dataset is available in Table \ref{table:litreview} along with the full list of references.

  \subsection{Literature review results}
    In total, our literature search found 3720 papers published between 2005 and 2014, and the number has increased substantially since 2005 (Fig. 1 in main text). 
    Most papers fit OU models to phylogenies with fewer than 100 taxa (mean = 166.97 $\pm$ 43.86, median = 58, Figure 2 in main text and Table \ref{table:litreview}). 
    The majority of papers fit OU models using R packages, particularly GEIGER and, although other uses are becoming more common, most papers use OU models in an effort to discern the ``best'' model of trait evolution or to control for phylogenetic non-independence (Table \ref{table:litreview}). 

\begin{landscape}
\begin{center}

\LTcapwidth=25cm
\begin{longtable}{p{6cm}llllll}
\caption[Literature review]
        {Details of the papers used in our literature review. For each paper we recorded the study taxon, the number of tips in the phylogeny used to fit the Ornstein Uhlenbeck (OU) model, how the authors used the OU model in the paper, and the statistical package (usually R) used to carry out the analyses. For a full reference list see below. Journal abbreviations: Am Nat = The American Naturalist, ELE = Ecology Letters, JEB = Journal of Evolutionary Biology, PRSB = Proceedings of the Royal Society B: Biological Sciences, PNAS = Proceedings of the National Academy of the USA. †Number of tips in the phylogeny the OU model was fitted to. Where there were multiple analyses in a paper we use the median number of tips.  All packages mentioned are R packages apart from BayesTraits, COMPARE, MATLAB, Mesquite, PAM, PDAP and PDTREE.}\\
\bigskip        
%\hline
\textbf{Paper}  &   \textbf{Year}    &   \textbf{Journal} &   \textbf{Taxon}   &   \textbf{Ntips}   &   \textbf{Use in paper}   &   \textbf{Stats/R package} \\
\hline
Hansen and Orzack 2005    &   2005    &   Evolution   &   insects &   15  &   other   &   OUCH precursor? \\
Edwards and Donoghue 2006     &   2006    &   Am Nat  &   plants  &   12  &   ancestral state reconstruction  &   COMPARE \\
Gvo\v{z}d\'{i}k and Damme 2006    &   2006    &   Evolution   &   amphibians  &   10  &   phylogenetic correction &   COMPARE \\
Halsey et al. 2006    &   2006    &   Am Nat  &   birds/mammals   &   90  &   phylogenetic correction &   Custom code \\
Ives and Godfray 2006 &   2006    &   Am Nat  &   insects &   8   &   phylogenetic signal &   MATLAB  \\
Valiente-Banuet et al. 2006   &   2006    &   PNAS    &   plants  &   47  &   model of evolution  &   ? OUCH    \\
Clabaut et al. 2007   &   2007    &   Evolution   &   fish    &   45  &   phylogenetic correction &   APE \\
Gomez and Th\'{e}ry 2007  &   2007    &   Am Nat  &   birds   &   40  &   model of evolution  &   OUCH    \\
Hipp 2007 &   2007    &   Evolution   &   plants  &   53  &   model of evolution  &   BayesTraits \\
Rezende et al. 2007   &   2007    &   Nature  &   plants/insects  &   ?   &   phylogenetic signal &   ? OUCH    \\
Spoor et al. 2007 &   2007    &   PNAS    &   mammals &   210 &   phylogenetic correction &   PDAP    \\
Stuart‐Fox et al. 2007    &   2007    &   Am Nat  &   reptiles    &   21  &   phylogenetic correction &   COMPARE \\
Buchwalter et al. 2008    &   2008    &   PNAS    &   insects &   21  &   phylogenetic correction &   MATLAB  \\
Dumont and Payseur 2008   &   2008    &   Evolution   &   mammals &   13  &   model of evolution  &   OUCH    \\
Hansen et al. 2008    &   2008    &   Evolution   &   mammals &   105 &   other   &   SLOUCH  \\
Smith et al. 2008 &   2008    &   Evolution   &   plants  &   15  &   phylogenetic correction &   APE \\
Warne and Charnov 2008    &   2008    &   Am Nat  &   reptiles    &   71  &   phylogenetic correction &   MATLAB  \\
Adams et al. 2009 &   2009    &   PRSB    &   amphibians  &   10  &   model of evolution  &   OUCH    \\
Addison et al. 2009   &   2009    &   PRSB    &   birds   &   23  &   phylogenetic correction &   MATLAB  \\
Agrawal et al. 2009   &   2009    &   PNAS    &   plants  &   53  &   model of evolution  &   GEIGER  \\
Bergmann et al. 2009  &   2009    &   Evolution   &   reptiles    &   38  &   model of evolution  &   OUCH    \\
Collar et al. 2009    &   2009    &   Evolution   &   fish    &   29  &   model of evolution  &   OUCH    \\
Gonzalez-Voyer et al. 2009    &   2009    &   PRSB    &   fish    &   39  &   phylogenetic correction &   APE/COMPARE \\
Goodman et al. 2009   &   2009    &   Evolution   &   reptiles    &   20  &   phylogenetic correction &   COMPARE \\
Huey et al. 2009  &   2009    &   PRSB    &   reptiles    &   70  &   phylogenetic correction &   MATLAB  \\
Kozak et al. 2009 &   2009    &   Evolution   &   amphibians  &   184 &   model of evolution  &   GEIGER  \\
Labra et al. 2009 &   2009    &   Am Nat  &   reptiles    &   83  &   other   &   SLOUCH  \\
Rezende et al. 2009   &   2009    &   ELE &   fish    &   116 &   model of evolution  &   PDTREE  \\
Swanson and Garland 2009  &   2009    &   Evolution   &   birds   &   44  &   phylogenetic correction &   MATLAB  \\
Van Buskirk 2009  &   2009    &   JEB &   amphibians  &   82  &   other   &   SLOUCH  \\
Burbrink and Pyron 2010   &   2010    &   Evolution   &   reptiles    &   29  &   model of evolution  &   GEIGER  \\
Cooper and Purvis 2010    &   2010    &   Am Nat  &   mammals &   45  &   model of evolution  &   GEIGER  \\
Edwards and Smith 2010    &   2010    &   PNAS    &   plants  &   300 &   model of evolution  &   OUCH    \\
Harmon et al. 2010    &   2010    &   Evolution   &   multiple    &   17  &   model of evolution  &   GEIGER  \\
Helmus et al. 2010    &   2010    &   ELE &   zooplankton &   15  &   phylogenetic correction &   ?   \\
Kalinka et al. 2010   &   2010    &   Nature  &   insects &   6   &   model of evolution  &   OUCH    \\
Kozak and Wiens 2010b &   2010    &   Am Nat  &   amphibians  &   84  &   model of evolution/ &   GEIGER/ \\
    &       &       &       &       &   ancestral state reconstruction  &   OUCH    \\
Kozak and Wiens 2010a &   2010    &   ELE &   amphibians  &   11  &   model of evolution  &   OUCH    \\
Ord et al. 2010   &   2010    &   Evolution   &   reptiles    &   16  &   other   &   SLOUCH  \\
Price et al. 2010 &   2010    &   Evolution   &   fish    &   122 &   model of evolution  &   GEIGER  \\
Slater et al. 2010    &   2010    &   PRSB    &   mammals &   84  &   model of evolution  &   OUCH    \\
Angielczyk et al. 2011    &   2011    &   Evolution   &   reptiles    &   8   &   model of evolution  &   OUCH    \\
Benesh et al. 2011    &   2011    &   Evolution   &   helminths   &   310 &   model of evolution  &   OUCH    \\
Collar et al. 2011    &   2011    &   Evolution   &   reptiles    &   37  &   model of evolution  &   Brownie \\
Derryberry et al. 2011    &   2011    &   Evolution   &   birds   &   285 &   model of evolution  &   GEIGER  \\
Galvan and Moller 2011    &   2011    &   JEB &   birds   &   323 &   phylogenetic correction &   COMPARE \\
Gonzalez-Voyer and Kolm 2011  &   2011    &   JEB &   fish    &   49  &   model of evolution  &   GEIGER  \\
Ord et al. 2011   &   2011    &   Evolution   &   multiple    &   23  &   phylogenetic signal &   SLOUCH  \\
Oufiero et al. 2011   &   2011    &   Evolution   &   reptiles    &   106 &   phylogenetic correction &   SLOUCH  \\
Perez et al. 2011 &   2011    &   JEB &   mammals &   29  &   phylogenetic correction &   APE \\
Raia and Meiri 2011   &   2011    &   Evolution   &   mammals &   842 &   model of evolution  &   MOTMOT  \\
Rosas-Guerrero et al. 2011    &   2011    &   Evolution   &   plants  &   20  &   phylogenetic correction &   APE \\
Setiadi et al. 2011   &   2011    &   Am Nat  &   amphibians  &   22  &   model of evolution  &   OUCH    \\
Smith et al. 2011 &   2011    &   Evolution   &   reptiles    &   15  &   model of evolution  &   GEIGER  \\
Tulli et al. 2011 &   2011    &   JEB &   reptiles    &   29  &   phylogenetic correction &   ?   \\
Turbill et al. 2011   &   2011    &   PRSB    &   mammals &   19  &   phylogenetic correction &   GEIGER  \\
Valido et al. 2011    &   2011    &   JEB &   plants  &   111 &   phylogenetic correction &   APE \\
Wiens et al. 2011 &   2011    &   ELE &   amphibians  &   337 &   model of evolution/ &   GEIGER/ \\
    &       &       &       &       &   phylogenetic correction &   COMPARE \\
Weir and Wheatcroft 2011  &   2011    &   PRSB    &   birds   &   232 &   model of evolution  &   GEIGER  \\
Beaulieu et al. 2012  &   2012    &   Evolution   &   plants  &   590 &   model of evolution  &   OUwie   \\
Betancur-R et al. 2012    &   2012    &   ELE &   fish    &   123 &   model of evolution  &   GEIGER  \\
Blankers et al. 2012  &   2012    &   JEB &   amphibians  &   189 &   phylogenetic correction &   GEIGER  \\
Boettiger et al. 2012 &   2012    &   Evolution   &   reptiles    &   23  &   ancestral state reconstruction  &   OUCH    \\
Burbrink et al. 2012  &   2012    &   PRSB    &   multiple    &   41  &   model of evolution  &   Custom code \\
Calosi et al. 2012    &   2012    &   JEB &   insects &   25  &   phylogenetic correction &   MATLAB  \\
Claramunt et al. 2012a    &   2012    &   Am Nat  &   birds   &   290 &   model of evolution  &   GEIGER  \\
Claramunt et al. 2012b    &   2012    &   PRSB    &   birds   &   282 &   model of evolution  &   GEIGER  \\
Davis et al. 2012 &   2012    &   JEB &   insects &   53  &   phylogenetic correction &   SLOUCH  \\
Diniz-Filho et al. 2012   &   2012    &   Evolution   &   mammals &   209 &   other   &   PAM \\
Fusco et al. 2012 &   2012    &   Evolution   &   trilobites  &   60  &   model of evolution  &   ?   \\
Gomez-Mestre et al. 2012  &   2012    &   Evolution   &   amphibians  &   720 &   phylogenetic correction &   APE \\
Ingram et al. 2012    &   2012    &   JEB &   food webs   &   20  &   model of evolution  &   GEIGER  \\
Kellermann et al. 2012a   &   2012    &   Evolution   &   insects &   94  &   phylogenetic correction &   SLOUCH  \\
Kellermann et al. 2012b   &   2012    &   PNAS    &   insects &   94  &   phylogenetic correction &   SLOUCH  \\
Nogueira et al. 2012  &   2012    &   JEB &   plants  &   105 &   phylogenetic signal &   GEIGER  \\
Ord 2012  &   2012    &   JEB &   reptiles    &   32  &   ancestral state reconstruction  &   SLOUCH  \\
Pearse and Hipp 2012  &   2012    &   Evolution   &   plants  &   56  &   phylogenetic correction &   SLOUCH  \\
Pellissier et al. 2012    &   2012    &   JEB &   insects &   83  &   model of evolution  &   ? APE \\
Price et al. 2012 &   2012    &   Evolution   &   fish    &   50  &   model of evolution  &   GEIGER  \\
Sallan and Friedman 2012  &   2012    &   PRSB    &   fish    &   100 &   model of evolution  &   GEIGER  \\
Santana et al. 2012   &   2012    &   Evolution   &   mammals &   85  &   model of evolution  &   OUCH    \\
Schmerler et al. 2012 &   2012    &   PRSB    &   plants  &   88  &   phylogenetic correction &   nlme    \\
Smith 2012    &   2012    &   Evolution   &   birds   &   42  &   phylogenetic correction &   APE/nlme    \\
Sookias et al. 2012   &   2012    &   PRSB    &   multiple    &   43  &   model of evolution  &   GEIGER  \\
Stireman et al. 2012  &   2012    &   JEB &   insects &   24  &   model of evolution  &   GEIGER  \\
Voje and Hansen 2012  &   2012    &   Evolution   &   insects &   30  &   phylogenetic correction &   SLOUCH  \\
Weir et al. 2012  &   2012    &   Evolution   &   birds   &   232 &   model of evolution  &   GEIGER  \\
Arbour and L\'{o}pez-Fern\'{a}ndez 2013   &   2013    &   PRSB    &   fish    &   27  &   model of evolution  &   OUCH    \\
Benesh et al. 2013    &   2013    &   Am Nat  &   helminths   &   143 &   phylogenetic correction &   APE \\
Blackburn et al. 2013 &   2013    &   Evolution   &   amphibians  &   18  &   model of evolution  &   GEIGER  \\
Christin et al. 2013  &   2013    &   PNAS    &   plants  &   545 &   model of evolution  &   GEIGER/OUCH \\
Fr\'{e}d\'{e}rich et al. 2013 &   2013    &   Am Nat  &   fish    &   208 &   model of evolution  &   OUwie   \\
Friedman et al. 2013  &   2013    &   Evolution   &   birds   &   15  &   model of evolution  &   OUCH    \\
Guerrero et al. 2013  &   2013    &   PNAS    &   plants/reptiles &   49  &   ancestral state reconstruction  &   GEIGER/ \\
    &       &       &       &       &       &   COMPARE \\
Hertz et al. 2013 &   2013    &   Evolution   &   reptiles    &   100 &   model of evolution  &   GEIGER  \\
Hossie et al. 2013    &   2013    &   JEB &   amphibians/ &   104 &   phylogenetic correction &   GEIGER  \\
    &       &       &   reptiles    &       &       &       \\
Knope and Scales 2013 &   2013    &   JEB &   fish    &   26  &   model of evolution  &   OUCH    \\
Kostikova et al. 2013 &   2013    &   Am Nat  &   plants  &   68  &   model of evolution  &   OUwie   \\
Lambert and Wiens 2013    &   2013    &   Evolution   &   reptiles    &   117 &   ancestral state reconstruction  &   GEIGER  \\
Lapiedra et al. 2013  &   2013    &   PRSB    &   birds   &   154 &   model of evolution  &   OUwie   \\
Litsios et al. 2013   &   2013    &   Evolution   &   plants  &   382 &   model of evolution  &   OUwie   \\
L\'{o}pez-Fern\'{a}ndez et al. 2013   &   2013    &   Evolution   &   fish    &   135 &   model of evolution  &   GEIGER  \\
Machac et al. 2013    &   2013    &   Evolution   &   mammals &   231 &   model of evolution  &   GEIGER  \\
Mahler et al. 2013    &   2013    &   Science &   reptiles    &   100 &   convergent evolution    &   SURFACE \\
Maia et al. 2013  &   2013    &   PNAS    &   birds   &   47  &   model of evolution  &   OUwie   \\
Mirceta et al. 2013   &   2013    &   Science &   mammals &   130 &   phylogenetic correction/    &   Mesquite/   \\
    &       &       &       &       &   ancestral state reconstruction  &   MATLAB  \\
Moen et al. 2013  &   2013    &   PRSB    &   amphibians  &   44  &   convergent evolution    &   GEIGER  \\
P\'{e}rez i de Lanuza et al. 2013 &   2013    &   JEB &   reptiles    &   42  &   ancestral state reconstruction  &   GEIGER  \\
Pienaar et al. 2013   &   2013    &   ELE &   birds   &   382 &   model of evolution  &   SLOUCH  \\
Quintero and Wiens 2013   &   2013    &   ELE &   multiple    &   500 &   ancestral state reconstruction  &   GEIGER/COMPARE  \\
Ryan and Shaw 2013    &   2013    &   PRSB    &   mammals &   34  &   phylogenetic correction &   MATLAB  \\
Seddon et al. 2013    &   2013    &   PRSB    &   birds   &   153 &   model of evolution  &   ? GEIGER  \\
Tanabe and Sota 2013  &   2013    &   Evolution   &   millipedes  &   84  &   phylogenetic correction &   APE \\
Voje et al. 2013  &   2013    &   JEB &   fish    &   87  &   other   &   SLOUCH  \\
Wiens et al. 2013 &   2013    &   Evolution   &   reptiles    &   117 &   ancestral state reconstruction  &   GEIGER  \\
\hline
\label{table:litreview}
\end{longtable}
\end{center}
\end{landscape}
   
\subsection{Literature review references}
\label{section:refs} 
\indent
Adams, D. C., C. M. Berns, K. H. Kozak, and J. J. Wiens. 2009. Are rates of species diversification correlated with rates of morphological evolution? Proceedings of the Royal Society B: Biological Sciences 276:2729-2738.\\
Addison, B., K. C. Klasing, W. D. Robinson, S. H. Austin, and R. E. Ricklefs. 2009. Ecological and life-history factors influencing the evolution of maternal antibody allocation: a phylogenetic comparison. Proceedings of the Royal Society B: Biological Sciences 276:3979-3987.\\
Agrawal, A. A., M. Fishbein, R. Halitschke, A. P. Hastings, D. L. Rabosky, and S. Rasmann. 2009. Evidence for adaptive radiation from a phylogenetic study of plant defenses. Proceedings of the National Academy of Sciences 106:18067-18072.\\
Angielczyk, K. D., C. R. Feldman, and G. R. Miller. 2011. Adaptive evolution of plastron shape in emydine turtles. Evolution 65:377-394.\\
Arbour, J. H., and H. L\'{o}pez-Fern\'{a}ndez. 2013. Ecological variation in South American geophagine cichlids arose during an early burst of adaptive morphological and functional evolution. Proceedings of the Royal Society B: Biological Sciences 280.\\
Beaulieu, J. M., D.-C. Jhwueng, C. Boettiger, and B. C. O’Meara. 2012. Modeling stabilizing selection: expanding the Ornstein-Uhlenbeck model of adaptive evolution. Evolution 66:2369-2383.\\
Benesh, D. P., J. C. Chubb, and G. A. Parker. 2011. Exploitation of the same trophic link favors convergence of larval life-history strategies in complex life cycle helminths. Evolution 65:2286-2299.\\
Benesh, D. P., J. C. Chubb, and G. A. Parker. 2013. Complex life cycles: why refrain from growth before reproduction in the adult niche? The American Naturalist 181:39-51.\\
Bergmann, P. J., J. J. Meyers, and D. J. Irschick. 2009. Directional evolution of stockiness coevolves with ecology and locomotion in lizards. Evolution 63:215-227.\\
Betancur-R, R., G. Ortí, A. M. Stein, A. P. Marceniuk, and R. Alexander Pyron. 2012. Apparent signal of competition limiting diversification after ecological transitions from marine to freshwater habitats. Ecology Letters 15:822-830.\\
Blackburn, D. C., C. D. Siler, A. C. Diesmos, J. A. McGuire, D. C. Cannatella, and R. M. Brown. 2013. An adaptive radiation of frogs in a Southeat Asian island archipelago. Evolution 67:2631–2646.\\
Blankers, T., D. C. Adams, and J. J. Wiens. 2012. Ecological radiation with limited morphological diversification in salamanders. Journal of evolutionary biology 25:634-646.\\
Boettiger, C., G. Coop, and P. Ralph. 2012. Is your phylogeny informative? Measuring the power of comparative methods. Evolution 66:2240-2251.\\
Buchwalter, D. B., D. J. Cain, C. A. Martin, L. Xie, S. N. Luoma, and T. Garland. 2008. Aquatic insect ecophysiological traits reveal phylogenetically based differences in dissolved cadmium susceptibility. Proceedings of the National Academy of Sciences 105:8321-8326.\\
Burbrink, F. T., X. Chen, E. A. Myers, M. C. Brandley, and R. A. Pyron. 2012. Evidence for determinism in species diversification and contingency in phenotypic evolution during adaptive radiation. Proceedings of the Royal Society B: Biological Sciences 279:4817-4826.\\
Burbrink, F. T., and R. A. Pyron. 2010. How does ecological opportunity influence rates of speciation, extinction, and morphological diversification in New World ratsnakes (tribe Lampropeltini)? Evolution 64:934-943.\\
Calosi, P., D. T. Bilton, J. I. Spicer, W. C. E. P. Verberk, A. Atfield, and T. Garland. 2012. The comparative biology of diving in two genera of European Dytiscidae (Coleoptera). Journal of Evolutionary Biology 25:329-341.\\
Christin, P.-A., C. P. Osborne, D. S. Chatelet, J. T. Columbus, G. Besnard, T. R. Hodkinson, L. M. Garrison, M. S. Vorontsova, and E. J. Edwards. 2013. Anatomical enablers and the evolution of C4 photosynthesis in grasses. Proceedings of the National Academy of Sciences 110:1381-1386.\\
Clabaut, C., P. M. E. Bunje, W. Salzburger, and A. Meyer. 2007. Geometric morphometric analyses provide evidence for the adaptive character of the Tanganyikan cichlid fish radiations. Evolution 61:560-578.\\
Claramunt, S., E. P. Derryberry, R. T. Brumfield, and J. V. Remsen Jr. 2012a. Ecological opportunity and diversification in a continental radiation of birds: climbing adaptations and cladogenesis in the Furnariidae. The American Naturalist 179:649-666.\\
Claramunt, S., E. P. Derryberry, J. V. Remsen, and R. T. Brumfield. 2012b. High dispersal ability inhibits speciation in a continental radiation of passerine birds. Proceedings of the Royal Society B: Biological Sciences 279:1567-1574.\\
Collar, D. C., B. C. O'Meara, P. C. Wainwright, and T. J. Near. 2009. Piscivory limits diversification of feeding morphology in centrarchid fishes. Evolution 63:1557-1573.\\
Collar, D. C., J. A. Schulte Ii, and J. B. Losos. 2011. Evolution of extreme body size disparity in monitor lizards (Varanus). Evolution 65:2664-2680.\\
Cooper, N., and A. Purvis. 2010. Body size evolution in mammals: complexity in tempo and mode. The American Naturalist 175:727-738.\\
Davis, R. B., J. Javois, J. Pienaar, E. Õunap, and T. Tammaru. 2012. Disentangling determinants of egg size in the Geometridae (Lepidoptera) using an advanced phylogenetic comparative method. Journal of Evolutionary Biology 25:210-219.\\
Derryberry, E. P., S. Claramunt, G. Derryberry, R. T. Chesser, J. Cracraft, A. Aleixo, J. Pérez-Emán, J. J. V. Remsen, and R. T. Brumfield. 2011. Lineage diversification and morphological evolution in a large-scale continental radiation: the Neotropical ovenbirds and woodcreepers (Aves: Furnariidae). Evolution 65:2973-2986.\\
Diniz-Filho, J. A. F., T. F. Rangel, T. Santos, and L. Mauricio Bini. 2012. Exploring patterns of interspecific variation in quantitative traits using sequential phylogenetic eigenvector regressions. Evolution 66:1079-1090.\\
Dumont, B. L., and B. A. Payseur. 2008. Evolution of the genomic rate of recombination in mammals Evolution 62:276-294.\\
Edwards, E. J., and M. J. Donoghue. 2006. Pereskia and the origin of the cactus life-form. American Naturalist 167:777-793.\\
Edwards, E. J., and S. A. Smith. 2010. Phylogenetic analyses reveal the shady history of C4 grasses. Proceedings of the National Academy of Sciences 107:2532-2537.\\
Fr\'{e}d\'{e}rich, B., L. Sorenson, F. Santini, G. J. Slater, and M. E. Alfaro. 2013. Iterative ecological radiation and convergence during the evolutionary history of damselfishes (Pomacentridae). The American Naturalist 181:94-113.\\
Friedman, N. R., K. J. McGraw, and K. E. Omland. 2013. Evolution of carotenoid pigmentation in caciques and meadowlarks (Icteridae): repeated gains of red plumage coloration by carotenoid C4-oxygenation. Evolution in press.\\
Fusco, G., J. T. Garland, G. Hunt, and N. C. Hughes. 2012. Developmental trait evolution in trilobites. Evolution 66:314-329.\\
Galvan, I., and A. P. Moller. 2011. Brain size and the expression of pheomelanin-based colour in birds. Journal of Evolutionary Biology 24:999-1006.\\
Gomez, D., and M. Th\'{e}ry. 2007. Simultaneous crypsis and conspicuousness in color patterns: comparative analysis of a neotropical rainforest bird community. The American Naturalist 169:S42-S61.\\
Gomez-Mestre, I., R. A. Pyron, and J. J. Wiens. 2012. Phylogenetic analyses reveal unexpected patterns in the evolution of reproductive modes in frogs. Evolution 66:3687-3700.\\
Gonzalez-Voyer, A., and N. Kolm. 2011. Rates of phenotypic evolution of ecological characters and sexual traits during the Tanganyikan cichlid adaptive radiation. Journal of Evolutionary Biology 24:2378-2388.\\
Gonzalez-Voyer, A., S. Winberg, and N. Kolm. 2009. Social fishes and single mothers: brain evolution in African cichlids. Proceedings of the Royal Society B: Biological Sciences 276:161-167.\\
Goodman, B. A., S. C. Hudson, J. L. Isaac, and L. Schwarzkopf. 2009. The evolution of body shape in response to habitat: is reproductive output reduced in flat lizards? Evolution 63:1279-1291.\\
Guerrero, P. C., M. Rosas, M. T. K. Arroyo, and J. J. Wiens. 2013. Evolutionary lag times and recent origin of the biota of an ancient desert (Atacama–Sechura). Proceedings of the National Academy of Sciences 110:11469-11474.\\
Gvo\v{z}d\'{i}k, L., and R. V. Damme. 2006. Triturus newts defy the running-swimming dilemma. Evolution 60:2110-2121.\\
Halsey, L. G., P. J. Butler, and T. M. Blackburn. 2006. A phylogenetic analysis of the allometry of diving. The American Naturalist 167:276-287.\\
Hansen, T. F., and S. H. Orzack. 2005. Assessing current adaptation and phylogenetic inertia as explanations of trait evolution: the need for controlled comparisons. Evolution 59:2063-2072.\\
Hansen, T. F., J. Pienaar, and S. H. Orzack. 2008. A comparative method for studying adaptation to a randomly evolving environment. Evolution 62:1965-1977.\\
Harmon, L. J., J. B. Losos, T. Jonathan Davies, R. G. Gillespie, J. L. Gittleman, W. Bryan Jennings, K. H. Kozak, M. A. McPeek, F. Moreno Roark, and T. J. Near. 2010. Early bursts of body size and shape evolution are rare in comparative data. Evolution 64:2385-2396.\\
Helmus, M. R., W. Keller, M. J. Paterson, N. D. Yan, C. H. Cannon, and J. A. Rusak. 2010. Communities contain closely related species during ecosystem disturbance. Ecology Letters 13:162-174.\\
Hertz, P. E., Y. Arima, A. Harrison, R. B. Huey, J. B. Losos, and R. E. Glor. 2013. Asynchronous evolution of physiology and morphology in Anolis lizards. Evolution 67:2101-2113.\\
Hipp, A. L. 2007. Nonuniform processes of chromosome evolution in sedges (Carex: Cyperaceae). Evolution 61:2175-2194.\\
Hossie, T. J., C. Hassall, W. Knee, and T. N. Sherratt. 2013. Species with a chemical defence, but not chemical offence, live longer. Journal of Evolutionary Biology 26:1598-1602.\\
Huey, R. B., C. A. Deutsch, J. J. Tewksbury, L. J. Vitt, P. E. Hertz, H. J. \'{A}lvarez P\'{e}rez, and T. Garland. 2009. Why tropical forest lizards are vulnerable to climate warming. Proceedings of the Royal Society of London Series B-Biological Sciences 276:1939-1948.\\
Ingram, T., L. J. Harmon, and J. B. Shurin. 2012. When should we expect early bursts of trait evolution in comparative data? Predictions from an evolutionary food web model. Journal of Evolutionary Biology 25:1902-1910.\\
Ives, A. R., and H. C. J. Godfray. 2006. Phylogenetic analysis of trophic associations. The American Naturalist 168:E1-E14.\\
Kalinka, A. T., K. M. Varga, D. T. Gerrard, S. Preibisch, D. L. Corcoran, J. Jarrells, U. Ohler, C. M. Bergman, and P. Tomancak. 2010. Gene expression divergence recapitulates the developmental hourglass model. Nature 468:811-814.\\
Kellermann, V., V. Loeschcke, A. A. Hoffmann, T. N. Kristensen, C. Fløjgaard, J. R. David, J.-C. Svenning, and J. Overgaard. 2012a. Phylogenetic constraints in key functional traits behind species' climate niches: patterns of desiccation and cold resistance across 95 Drosophila species. Evolution 66:3377-3389.\\
Kellermann, V., J. Overgaard, A. A. Hoffmann, C. Fløjgaard, J.-C. Svenning, and V. Loeschcke. 2012b. Upper thermal limits of Drosophila are linked to species distributions and strongly constrained phylogenetically. Proceedings of the National Academy of Sciences 109:16228-16233.\\
Knope, M. L., and J. A. Scales. 2013. Adaptive morphological shifts to novel habitats in marine sculpin fishes. Journal of Evolutionary Biology 26:472-482.\\
Kostikova, A., G. Litsios, N. Salamin, and P. B. Pearman. 2013. Linking life-history traits, ecology, and niche breadth evolution in North American eriogonoids (Polygonaceae). The American Naturalist 182:760-774.\\
Kozak, K. H., R. W. Mendyk, and J. J. Wiens. 2009. Can parallel diversification occur in sympatry? Repeated patterns of body-size evolution in coexisting clades of North American salamanders. Evolution 63:1769-1784.\\
Kozak, K. H., and J. J. Wiens. 2010a. Accelerated rates of climatic-niche evolution underlie rapid species diversification. Ecology Letters 13:1378-1389.\\
Kozak, K. H., and J. J. Wiens. 2010b. Niche conservatism drives elevational diversity patterns in Appalachian salamanders. The American Naturalist 176:40-54.\\
Labra, A., J. Pienaar, and T. F. Hansen. 2009. Evolution of thermal physiology in Liolaemus lizards: adaptation, phylogenetic inertia, and niche tracking. The American Naturalist 174:204-220.\\
Lambert, S. M., and J. J. Wiens. 2013. Evolution of viviparity: a phylogenetic test of the cold-climate hypothesis in phrynosomatid lizards. Evolution 67:2614-2630.\\
Lapiedra, O., D. Sol, S. Carranza, and J. M. Beaulieu. 2013. Behavioural changes and the adaptive diversification of pigeons and doves. Proceedings of the Royal Society B: Biological Sciences 280.\\
Litsios, G., R. O. Wüest, A. Kostikova, F. Forest, C. Lexer, H. P. Linder, P. B. Pearman, N. E. Zimmermann, and N. Salamin. 2013. Effects of a fire response trait on diversification in replicated radiations. Evolution in press.\\
L\'{o}pez-Fern\'{a}ndez, H., J. H. Arbour, K. O. Winemiller, and R. L. Honeycutt. 2013. Testing for ancient adaptive radiations in Neotropical cichlid fishes. Evolution 67:1321-1337.\\
Machac, A., D. Storch, and J. J. Wiens. 2013. Ecological causes of decelerating diversification in carnivoran mammals. Evolution 67:2423–2433.\\
Mahler, D. L., T. Ingram, L. J. Revell, and J. B. Losos. 2013. Exceptional convergence on the macroevolutionary landscape in island lizard radiations. Science 341:292-295.\\
Maia, R., D. R. Rubenstein, and M. D. Shawkey. 2013. Key ornamental innovations facilitate diversification in an avian radiation. Proceedings of the National Academy of Sciences in press.\\
Mirceta, S., A. V. Signore, J. M. Burns, A. R. Cossins, K. L. Campbell, and M. Berenbrink. 2013. Evolution of mammalian diving capacity traced by myoglobin net surface charge. Science 340:1234192.\\
Moen, D. S., D. J. Irschick, and J. J. Wiens. 2013. Evolutionary conservatism and convergence both lead to striking similarity in ecology, morphology and performance across continents in frogs. Proceedings of the Royal Society B: Biological Sciences 280:20132156.\\
Nogueira, A., P. J. Rey, and L. G. Lohmann. 2012. Evolution of extrafloral nectaries: adaptive process and selective regime changes from forest to savanna. Journal of Evolutionary Biology 25:2325-2340.\\
Ord, T. J. 2012. Historical contingency and behavioural divergence in territorial Anolis lizards. Journal of Evolutionary Biology 25:2047-2055.\\
Ord, T. J., L. King, and A. R. Young. 2011. Contrasting theory with the empirical data of species recognition. Evolution 65:2572-2591.\\
Ord, T. J., J. A. Stamps, and J. B. Losos. 2010. Adaptation and plasticity of animal communication in fluctuating environments. Evolution 64:3134-3148.\\
Oufiero, C. E., G. E. A. Gartner, S. C. Adolph, and T. Garland. 2011. Latitudinal and climatic variation in body size and dorsal scale counts in Sceloporus lizards: a phylogenetic perspective. Evolution 65:3590-3607.\\
Pearse, I. S., and A. L. Hipp. 2012. Global patterns of leaf defenses in oak species. Evolution 66:2272-2286.\\
Pellissier, L., S. Rasmann, G. Litsios, K. Fiedler, A. Dubuis, J. Pottier, and A. Guisan. 2012. High host-plant nitrogen content: a prerequisite for the evolution of ant–caterpillar mutualism? Journal of Evolutionary Biology 25:1658-1666.\\
P\'{e}rez i de Lanuza, G., E. Font, and J. L. Monterde. 2013. Using visual modelling to study the evolution of lizard coloration: sexual selection drives the evolution of sexual dichromatism in lacertids. Journal of Evolutionary Biology 26:1826-1835.\\
Perez, S. I., J. Klaczko, G. Rocatti, and S. F. Dos Reis. 2011. Patterns of cranial shape diversification during the phylogenetic branching process of New World monkeys (Primates: Platyrrhini). Journal of Evolutionary Biology 24:1826-1835.\\
Pienaar, J., A. Ilany, E. Geffen, and Y. Yom-Tov. 2013. Macroevolution of life‐history traits in passerine birds: adaptation and phylogenetic inertia. Ecology Letters 16:571–576.\\
Price, S. A., J. J. Tavera, T. J. Near, and P. Wainwright. 2012. Elevated rates of morphological and functional diversification in reef-dwelling haemulid fishes. Evolution 67:417–428.\\
Price, S. A., P. C. Wainwright, D. R. Bellwood, E. Kazancioglu, D. C. Collar, and T. J. Near. 2010. Functional innovations and morphological diversification in parrotfish. Evolution 64:3057–3068.\\
Quintero, I., and J. J. Wiens. 2013. Rates of projected climate change dramatically exceed past rates of climatic niche evolution among vertebrate species. Ecology letters 16:1095-1103.\\
Raia, P., and S. Meiri. 2011. The tempo and mode of evolution: body sizes of island mammals. Evolution 65:1927-1934.\\
Rezende, E. L., E. M. Albert, M. A. Fortuna, and J. Bascompte. 2009. Compartments in a marine food web associated with phylogeny, body mass, and habitat structure. Ecology Letters 12:779-788.\\
Rezende, E. L., J. E. Lavabre, P. R. Guimarães, P. Jordano, and J. Bascompte. 2007. Non-random coextinctions in phylogenetically structured mutualistic networks. Nature 448:925-928.\\
Rosas-Guerrero, V., M. Quesada, W. S. Armbruster, R. Pérez-Barrales, and S. D. Smith. 2011. Influence of pollination specialization and breeding system on floral integration and phenotypic variation in Ipomoea. Evolution 65:350-364.\\
Ryan, T. M., and C. N. Shaw. 2013. Trabecular bone microstructure scales allometrically in the primate humerus and femur. Proceedings of the Royal Society B: Biological Sciences 280.\\
Sallan, L. C., and M. Friedman. 2012. Heads or tails: staged diversification in vertebrate evolutionary radiations. Proceedings of the Royal Society B: Biological Sciences 279:2025-2032.\\
Santana, S. E., I. R. Grosse, and E. R. Dumont. 2012. Dietary hardness, loading behavior, and the evolution of skull form in bats. Evolution 66:2587-2598.\\
Schmerler, S. B., W. L. Clement, J. M. Beaulieu, D. S. Chatelet, L. Sack, M. J. Donoghue, and E. J. Edwards. 2012. Evolution of leaf form correlates with tropical–temperate transitions in Viburnum (Adoxaceae). Proceedings of the Royal Society B: Biological Sciences 279:3905-3913.\\
Seddon, N., C. A. Botero, J. A. Tobias, P. O. Dunn, H. E. A. MacGregor, D. R. Rubenstein, J. A. C. Uy, J. T. Weir, L. A. Whittingham, and R. J. Safran. 2013. Sexual selection accelerates signal evolution during speciation in birds. Proceedings of the Royal Society B: Biological Sciences 280.\\
Setiadi, M. I., J. A. McGuire, R. M. Brown, M. Zubairi, D. T. Iskandar, N. Andayani, J. Supriatna, and B. J. Evans. 2011. Adaptive radiation and ecological opportunity in Sulawesi and Philippine fanged frog (Limnonectes) communities. The American Naturalist 178:221-240.\\
Slater, G. J., S. A. Price, F. Santini, and M. E. Alfaro. 2010. Diversity versus disparity and the radiation of modern cetaceans. Proceedings of the Royal Society B: Biological Sciences 277:3097-3104.\\
Smith, K. L., L. J. Harmon, L. P. Shoo, and J. Melville. 2011. Evidence of constrained phenotypic evolution in a cryptic species complex of agamid lizards. Evolution 65:976-992.\\
Smith, N. D. 2012. Body mass and foraging ecology predict evolutionary patterns of skeletal pnematicity in the diverse "waterbird" clade. Evolution 66:1059-1078.\\
Smith, S. D., C. An\'{e}, and D. A. Baum. 2008. The role of pollinator shifts in the floral diversification of Iochroma (Solanaceae). Evolution 62:793-806.\\
Sookias, R. B., R. J. Butler, and R. B. J. Benson. 2012. Rise of dinosaurs reveals major body-size transitions are driven by passive processes of trait evolution. Proceedings of the Royal Society B: Biological Sciences 279:2180-2187.\\
Spoor, F., T. Garland, G. Krovitz, T. M. Ryan, M. T. Silcox, and A. Walker. 2007. The primate semicircular canal system and locomotion. Proceedings of the National Academy of Sciences 104:10808-10812.\\
Stireman, J. O., H. Devlin, and P. Abbot. 2012. Rampant host- and defensive phenotype-associated diversification in a goldenrod gall midge. Journal of Evolutionary Biology 25:1991-2004.\\
Stuart‐Fox, D., A. Moussalli, and M. J. Whiting. 2007. Natural selection on social signals: signal efficacy and the evolution of chameleon display coloration. The American Naturalist 170:916-930.\\
Swanson, D. L., and J. T. Garland. 2009. The evolution of high summit metabolism and cold tolerance in birds and its impact on present-day distributions. Evolution 63:184-194.\\
Tanabe, T., and T. Sota. 2013. Both male and female novel traits promote the correlated evolution of genitalia between the sexes in an arthropod. Evolution in press.\\
Tulli, M. J., V. Abdala, and F. B. Cruz. 2011. Relationships among morphology, clinging performance and habitat use in Liolaemini lizards. Journal of Evolutionary Biology 24:843-855.\\
Turbill, C., C. Bieber, and T. Ruf. 2011. Hibernation is associated with increased survival and the evolution of slow life histories among mammals. Proceedings of the Royal Society B: Biological Sciences 278:3355-3363.\\
Valido, A., H. M. Schaefer, and P. Jordano. 2011. Colour, design and reward: phenotypic integration of fleshy fruit displays. Journal of Evolutionary Biology 24:751-760.\\
Valiente-Banuet, A., A. V. Rumebe, M. Verdú, and R. M. Callaway. 2006. Modern Quaternary plant lineages promote diversity through facilitation of ancient Tertiary lineages. Proceedings of the National Academy of Sciences 103:16812-16817.\\
Van Buskirk, J. 2009. Getting in shape: adaptation and phylogenetic inertia in morphology of Australian anuran larvae. Journal of evolutionary biology 22:1326-1337.\\
Voje, K. L., and T. F. Hansen. 2012. Evolution of static allometries: adaptive change in allometric slopes of eye span in stalk-eyed flies. Evolution 67:453-467.\\
Voje, K. L., A. B. Mazzarella, T. F. Hansen, K. \o{O}stbye, T. Klepaker, A. Bass, A. Herland, K. M. Bærum, F. Gregersen, and L. A. V\o{o}llestad. 2013. Adaptation and constraint in a stickleback radiation. Journal of evolutionary biology 26:2396-2414.\\
Warne, R. W., and E. L. Charnov. 2008. Reproductive allometry and the size‐number trade‐off for lizards. The American Naturalist 172:E80-E98.\\
Weir, J. T., and D. Wheatcroft. 2011. A latitudinal gradient in rates of evolution of avian syllable diversity and song length. Proceedings of the Royal Society B: Biological Sciences 278:1713-1720.\\
Weir, J. T., D. J. Wheatcroft, and T. D. Price. 2012. The role of ecological constraint in driving the evolution of avian song frequency across a latitudinal gradient. Evolution 66:2773-2783.\\
Wiens, J. J., K. H. Kozak, and N. Silva. 2013. Diversity and niche evolution along aridity gradients in North American lizards (Phrynosomatidae). Evolution 67:1715-1728.\\
Wiens, J. J., R. A. Pyron, and D. S. Moen. 2011. Phylogenetic origins of local-scale diversity patterns and the causes of Amazonian megadiversity. Ecology Letters 14:643-652.

\end{document}